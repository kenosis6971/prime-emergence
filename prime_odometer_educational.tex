\documentclass[11pt]{article}

% --- Packages ---
\usepackage{amsmath, amssymb, amsthm} % math symbols, theorems
\usepackage{fullpage}                 % better margins
\usepackage{enumitem}                 % custom lists (for exercises)
\usepackage{hyperref}                 % hyperlinks
\usepackage{mathtools}                % extensions of amsmath
\usepackage{physics}                  % \abs, \norm, etc.
\usepackage{graphicx}                 % figures
\usepackage{tikz}                     % diagrams
\usepackage{multicol}                 % multi-column layout for exercises if needed
\usepackage{fancyhdr}                 % custom headers/footers
\usepackage{booktabs}                 % better tables
\usepackage{caption}                  % caption formatting
\usepackage{tcolorbox}                % colored boxes (for examples, exercises)

% --- Page style ---
\pagestyle{fancy}
\fancyhf{}                % clear all header/footers
\rfoot{\thepage}          % page number in lower right
\renewcommand{\headrulewidth}{0pt} % remove top line
\renewcommand{\footrulewidth}{0pt} % no bottom line

% --- Theorem styles ---
\theoremstyle{definition}
\newtheorem{definition}{Definition}[section]
\newtheorem{example}{Example}[section]
\newtheorem{exercise}{Exercise}[section]
\newtheorem{theorem}{Theorem}[section]
\newtheorem{corollary}{Corollary}[section]

% --- Custom environments ---
\newtcolorbox{studentnote}[1][]{
  colback=blue!5!white,
  colframe=blue!50!black,
  title=Note,#1}

\newtcolorbox{studentexercise}[1][]{
  colback=green!5!white,
  colframe=green!40!black,
  title=Exercise,#1}

\newtcolorbox{studentexample}[1][]{
  colback=orange!5!white,
  colframe=orange!85!black,
  title=Example,#1}

% --- Metadata ---
\title{The Prime Odometer: A Student-Friendly Introduction to Generative Prime Emergence}
\author{Kenneth Hilton}
\date{August 2025}

\begin{document}
\maketitle

\begin{abstract}
Prime numbers are the ``atoms of arithmetic,'' yet they are usually defined negatively: 
a number is prime if it has no divisors other than 1 and itself. 
Classical algorithms follow this mindset, marking off composites to leave primes behind. 
In this paper, we present the \emph{Prime Odometer}, a constructive and intuitive way 
of seeing primes emerge as inevitable ``carry events'' in a coordinate system called 
\emph{Prime Exponent Space} (PE-space). Instead of elimination, primes appear positively 
when the arithmetic structure can no longer redistribute. 

This student-oriented edition introduces the main ideas step by step, with examples, 
analogies, figures, and exercises, to make the theory accessible to undergraduate number 
theory students. Along the way, we contrast the Prime Odometer with the classical Sieve 
of Eratosthenes, provide proofs with explicit small-number cases, and explore why this 
reframing of primality as ``canonical carries'' is both mathematically rigorous and 
conceptually illuminating.
\end{abstract}

\section*{Notation Guide}
Throughout the paper we use several symbols and conventions that may be unfamiliar. 
This section introduces them with examples for clarity.

\begin{itemize}
  \item $\backslash$ : \textbf{Set difference}. 
  If $A = \{1,2,3\}$ and $B = \{2\}$, then $A \backslash B = \{1,3\}$.

  \item $\equiv$ : \textbf{Congruence modulo $p$}. 
  Writing $a \equiv b \pmod{p}$ means $p$ divides $(a-b)$. 
  For example, $17 \equiv 2 \pmod{5}$ since $17-2=15$ is divisible by 5.

  \item $\sim$ : \textbf{Asymptotic equivalence}. 
  We write $f(n) \sim g(n)$ as $n \to \infty$ when $\lim_{n \to \infty} f(n)/g(n) = 1$. 
  For instance, the Prime Number Theorem states $\pi(x) \sim \tfrac{x}{\log x}$.

  \item $\forall$ : \textbf{For all}. Example: $\forall n \geq 2, \, n$ has a unique factorization.

  \item $\exists$ : \textbf{There exists}. Example: $\exists$ infinitely many primes.

  \item $\sigma(n)$ : \textbf{Prime exponent vector}. 
  If $n = \prod p_i^{e_i}$, then $\sigma(n) = (e_1,e_2,e_3,\dots)$. 
  Example: $12 = 2^2 \cdot 3^1$, so $\sigma(12) = (2,1,0,0,\dots)$.

  \item $\nu_p(n)$ : \textbf{$p$-adic valuation}. 
  The highest power of $p$ dividing $n$. 
  Example: $\nu_2(40) = 3$ since $40 = 2^3 \cdot 5$.

  \item $\mathbb{N}_{\geq 2}$ : the set of integers $\{2,3,4,\dots\}$.
\end{itemize}

\vspace{1em}
\noindent
\textbf{Tip for students:} whenever you see a symbol you do not immediately recognize, 
refer back to this guide. New notation is always introduced with examples before being 
used in proofs or algorithms.

\section{Introduction}

Prime numbers are the fundamental building blocks of arithmetic. 
Every integer greater than one can be expressed uniquely as a product of primes, 
a fact known as the \emph{Fundamental Theorem of Arithmetic}. 
Despite this central role, primes have long appeared mysterious in their distribution: 
they seem to pop up unpredictably along the number line. 

Traditionally, primes are defined in a negative way:

\begin{quote}
\textbf{Definition (Classical).} A prime is an integer $p > 1$ with no divisors other than $1$ and $p$ itself.
\end{quote}

While correct, this definition feels unsatisfying. It says only what primes are \emph{not} 
(divisible by anything else), rather than what they \emph{are}. 

\subsection*{Historical Context}
Euclid, over two thousand years ago, proved that there are infinitely many primes. 
Centuries later, Eratosthenes devised his famous sieve, systematically eliminating multiples 
of each prime to reveal those that remain. Modern algorithms still reflect this eliminative 
mindset: primality tests probe for factors, sieves mark off composites, and analytic number 
theory describes global patterns without explaining local inevitability. 

For millennia, primes have thus been treated as the “survivors” left after composites are excluded. 

\subsection*{A New Perspective: Generative Primes}
In this paper we adopt a different, more constructive view: 
primes appear not as anomalies but as \emph{inevitable carry events} in a structural framework 
called \emph{Prime Exponent Space} (PE-space). 

The key idea is simple:

\begin{quote}
Primes are the moments when arithmetic can no longer redistribute across existing factors, 
and a new prime axis must be created.
\end{quote}

This event is directly analogous to the carry in positional notation. 

\subsection*{Analogy: The Odometer}
Consider an odometer or a counter in base 10:
\[
999 \longmapsto 1000.
\]
When you add one to 999, every digit “overflows,” and a new digit place must appear. 
The carry is not optional --- it is structurally forced by the rules of arithmetic. 

In PE-space, an analogous carry occurs not at every step, but exactly at the primes. 
At these special points, redistribution across all existing prime axes fails, and the odometer 
is forced to roll over into a new axis. The prime itself emerges as the one-hot vector 
that marks this structural inevitability. 

\subsection*{Goals of This Paper}
Our purpose in this student edition is threefold:
\begin{enumerate}
  \item To introduce \emph{Prime Exponent Space} clearly, with examples and diagrams.
  \item To explain the \emph{Prime Odometer Algorithm}, showing step by step how primes 
        emerge constructively rather than by elimination.
  \item To explore the mathematical and philosophical implications of this perspective, 
        contrasting it with classical sieves and highlighting its value for teaching and intuition.
\end{enumerate}

We build up the theory gradually. Each new piece of notation is introduced with a worked example. 
Exercises throughout invite you to test your understanding, and side notes offer historical 
and conceptual context. By the end, you will see primes not as mysterious exceptions but as 
structural necessities --- the canonical carries of arithmetic itself. 

\begin{studentnote}
\textbf{Roadmap.} 
Section 2 defines Prime Exponent Space with examples.  
Section 3 introduces smooth frontiers and pre-primes.  
Section 4 presents the Prime Odometer algorithm.  
Section 5 proves its correctness.  
Section 6 compares it to classical sieves.  
Section 7 explores implications and perspectives.  
Exercises are collected in Section 8, and the paper concludes in Section 9.  
Appendices provide pseudocode and extended notation.
\end{studentnote}

\section{Prime Exponent Space (PE-Space)}

The central tool of the Prime Odometer is a new coordinate system for the integers: 
\emph{Prime Exponent Space}, abbreviated PE-space. 
This representation is rooted in the \emph{Fundamental Theorem of Arithmetic}, which states 
that every integer greater than 1 can be written uniquely as a product of prime powers.

\begin{definition}[Prime Exponent Vector]
Let $p_1 = 2, p_2 = 3, p_3 = 5, \dots$ be the ordered primes. 
For each integer $n \geq 2$, write its unique prime factorization:
\[
n = \prod_{i \geq 1} p_i^{e_i(n)} ,
\]
where all but finitely many $e_i(n)$ are zero. 
The \emph{prime exponent vector} of $n$ is
\[
\sigma(n) = (e_1(n), e_2(n), e_3(n), \dots).
\]
\end{definition}

Thus every integer corresponds to a point in an infinite-dimensional lattice 
with nonnegative integer coordinates. 

\subsection{Basic Properties}
Prime exponent vectors make arithmetic simpler:
\begin{align*}
\sigma(ab) &= \sigma(a) + \sigma(b), \\\\
\sigma\!\left(\frac{a}{b}\right) &= \sigma(a) - \sigma(b),
\end{align*}
whenever $b$ divides $a$. In words: multiplication becomes vector addition, 
and division becomes subtraction.

\begin{studentexample}
Let $a = 12 = 2^2 \cdot 3^1$ and $b = 18 = 2^1 \cdot 3^2$.  
Then
\[
\sigma(a) = (2,1,0,0,\dots), \quad \sigma(b) = (1,2,0,0,\dots).
\]
Their product is $a \cdot b = 216 = 2^3 \cdot 3^3$, and indeed
\[
\sigma(ab) = (3,3,0,0,\dots) = \sigma(a) + \sigma(b).
\]
\end{studentexample}

\begin{studentnote}
\textbf{Why is this helpful?}  
On the number line, multiplication feels complicated, 
but in PE-space it reduces to simple vector addition. 
The price we pay is that the successor operation $n \mapsto n+1$ 
becomes more subtle --- and this is exactly where the Prime Odometer comes in. 
\end{studentnote}

\subsection{Primes as One-Hot Vectors}
In PE-space, primes have a distinctive form.  
Each prime $p_k$ corresponds to a \emph{one-hot vector}: all entries are zero except 
for a single $1$ in the $k$th position.

\begin{studentexample}
\begin{itemize}
  \item $2 = 2^1$, so $\sigma(2) = (1,0,0,0,\dots)$.
  \item $3 = 3^1$, so $\sigma(3) = (0,1,0,0,\dots)$.
  \item $5 = 5^1$, so $\sigma(5) = (0,0,1,0,\dots)$.
  \item $7 = 7^1$, so $\sigma(7) = (0,0,0,1,0,\dots)$.
\end{itemize}
\end{studentexample}

Composites, by contrast, have multiple nonzero entries.  
For example:
\[
12 = 2^2 \cdot 3^1 \quad \Rightarrow \quad \sigma(12) = (2,1,0,0,\dots).
\]

\subsection{Dimensional Viewpoint}
One way to think of PE-space is as a coordinate system where each prime defines an axis.  
Composites are points inside the span of existing axes.  
Primes, however, mark the \emph{first use of a new axis}.  
In this sense, primes are dimension-creating events.

\begin{studentnote}
When a new prime appears, the dimension of our representation grows. 
This is why we say primes are not just numbers, but structural milestones: 
they extend the dimensionality of number space. 
\end{studentnote}

\subsection*{Exercises (Set \#1)}
\begin{studentexercise}
Compute the prime exponent vectors for:
\[
20, \quad 30, \quad 64, \quad 210.
\]
(Hint: factor each number first.)
\end{studentexercise}

\begin{studentexercise}
Show that $\sigma(45) + \sigma(14) = \sigma(630)$. 
Verify this equality both by prime factorization and by vector addition.
\end{studentexercise}

\begin{studentexercise}
Which numbers less than 20 are represented by one-hot vectors in PE-space? 
What does this tell you?
\end{studentexercise}

\section{Smooth Frontiers and Pre-Primes}

Having defined PE-space, we now investigate how the sequence of natural numbers 
moves through it. The key idea is that the set of integers whose prime factors 
are bounded by some prime $p_k$ forms a kind of ``frontier.'' 
When we step past that frontier, a new prime dimension must be introduced. 

\begin{definition}[$p_k$-smooth semigroup]
Fix $k \geq 1$. Define
\[
S_k = \Bigl\{ \prod_{i=1}^k p_i^{e_i} : e_i \in \mathbb{N}_0 \Bigr\}.
\]
This is the set of integers whose prime factors are among the first $k$ primes 
($2,3,\dots,p_k$). We call such numbers \emph{$p_k$-smooth}.
\end{definition}

\begin{definition}[Pre-prime at level $k$]
A number $n \in S_k$ is called a \emph{pre-prime at level $k$} 
if its successor $n+1 \notin S_k$. 
In other words, $n$ is inside the frontier, but $n+1$ lies outside. 
\end{definition}

\begin{theorem}[Emergence Threshold]
The next prime $p_{k+1}$ is exactly the smallest integer not in $S_k$:
\[
p_{k+1} = \min\bigl(\mathbb{N}_{\geq 2} \setminus S_k \bigr).
\]
Equivalently, $p_{k+1}-1 \in S_k$ and $p_{k+1} \notin S_k$.
\end{theorem}

\begin{proof}[Sketch of proof]
If $n \notin S_k$, then $n$ has a prime factor $q > p_k$. 
The smallest such $n$ must be $q$ itself, which is prime, 
and hence $p_{k+1}$. By definition, $p_{k+1}-1$ uses only smaller primes, 
so $p_{k+1}-1 \in S_k$.
\end{proof}

\subsection{Worked Example: Frontier at $p_3=5$}
Let $k=3$, so $S_3$ consists of numbers whose prime factors are only $2,3,5$.  
This includes:
\[
2,3,4,5,6,8,9,10,12,15,16,18,20,24,25,27,30, \dots
\]

\begin{itemize}
  \item $6 = 2 \cdot 3$ belongs to $S_3$. Its successor $7$ is not in $S_3$, 
        because 7 requires a new prime factor.  
        Thus 6 is a pre-prime, and 7 is prime.  
  \item Similarly, $10 \in S_3$, but $11 \notin S_3$, so $10$ is a pre-prime 
        and $11$ is prime.  
  \item $12 \in S_3$, but $13 \notin S_3$, so $12$ is a pre-prime 
        and $13$ is prime.
\end{itemize}

Notice that just before each new prime, 
we find a number in $S_k$ that ``cannot step forward'' without leaving the frontier. 

\begin{studentnote}
This explains why we call these numbers \emph{pre-primes}: 
they are exactly the integers that immediately precede primes, 
marking the edge of smooth frontiers in PE-space. 
\end{studentnote}

\subsection{Canonical Carries}
The step $p_{k+1}-1 \mapsto p_{k+1}$ is not arbitrary. 
It is structurally forced: all smaller primes fail to absorb the successor, 
so the odometer ``rolls over'' into a new axis. 
We call this event a \emph{canonical carry}. 

\begin{studentexample}
At $n=6$, the next integer is 7.  
All smaller primes $2,3,5$ fail to divide 7.  
Thus the odometer is forced to carry into a new axis, creating the prime 7.  
In vector form:
\[
\sigma(6) = (1,1,0,0,\dots), \quad 
\sigma(7) = (0,0,0,1,0,\dots).
\]
\end{studentexample}

\begin{studentnote}
\textbf{Analogy.}  
Think again of base-10: when the digits $999$ ``overflow,'' 
the only option is to add a new digit.  
In PE-space, when no existing prime axis can absorb the increment, 
a new prime axis must appear. That is the birth of a prime. 
\end{studentnote}

\subsection*{Exercises (Set \#2)}
\begin{studentexercise}
List all elements of $S_2$ (the $3$-smooth numbers) up to 30. 
Which of them are pre-primes? What primes follow them?
\end{studentexercise}

\begin{studentexercise}
Show that 30 is a pre-prime in $S_3$.  
Which prime follows it?  
Write both $\sigma(30)$ and $\sigma(31)$.
\end{studentexercise}

\begin{studentexercise}
Verify that for each prime $p$, $p-1$ lies in the smooth semigroup 
defined by the previous primes. 
(Hint: factor $p-1$ into primes smaller than $p$.)
\end{studentexercise}

\section{The Prime Odometer Algorithm}

We are now ready to define the \emph{Prime Odometer} itself. 
This is an incremental algorithm that advances one integer at a time, 
updating a compact state. 
At each step $n \mapsto n+1$, the odometer either redistributes across 
known prime axes (composite case) or is forced to create a new axis (prime case). 

\subsection{Residue Counters and State Representation}

The algorithm keeps track of only a finite amount of information at each stage:

\begin{itemize}
  \item A list $P = \{p_1, p_2, \dots, p_k\}$ of all primes discovered so far, 
        sufficient to factor numbers up to the current bound.
  \item For each prime $p \in P$, a residue counter
        \[
        r_p(n) \equiv n \pmod{p}.
        \]
        This counter increments with $n$, and resets to $0$ exactly when $p \mid n$.
  \item (Optional) Higher-power residues $r_{p^t}(n) \equiv n \pmod{p^t}$, 
        which allow direct computation of valuations $\nu_p(n)$.
\end{itemize}

Thus the state at $n$ is finite and compact.  
We never need to re-factor numbers from scratch; 
instead, we update residues step by step.

\subsection{Successor Update Rule}

To advance from $n$ to $m = n+1$, the odometer performs:

\begin{enumerate}
  \item \textbf{Increment residues.}  
        For each $p \in P$, update
        \[
        r_p(m) = (r_p(n) + 1) \bmod p.
        \]
  \item \textbf{Check for divisibility.}  
        If $r_p(m) = 0$ for some $p \leq \sqrt{m}$, 
        then $p \mid m$.  
        Repeatedly check higher powers $p^t$ if needed to determine $\nu_p(m)$.
  \item \textbf{Prime case.}  
        If no $p \in P$ with $p \leq \sqrt{m}$ divides $m$, 
        then $m$ is prime.  
        Append $m$ to $P$, initialize $r_m(m) = 0$, 
        and record $\sigma(m)$ as a one-hot vector.
\end{enumerate}

\begin{studentnote}
This rule makes the Prime Odometer:
\begin{itemize}
  \item \emph{Generative:} builds each successor from the previous state.
  \item \emph{Deterministic:} no randomness or probabilistic tests.
  \item \emph{Factor-free:} never re-factors $m$ from scratch.
  \item \emph{Sequential:} primes emerge exactly in order.
\end{itemize}
\end{studentnote}

\subsection{Worked Example: From 5 to 7}

Suppose we are at $n=5$. The current prime list is $P = \{2,3\}$.

\begin{enumerate}
  \item Residues at $n=5$:  
        \[
        r_2(5) = 1, \quad r_3(5) = 2.
        \]
  \item Tick to $n=6$:  
        \[
        r_2(6) = 0, \quad r_3(6) = 0.
        \]
        Since both residues are zero, $6$ is divisible by $2$ and $3$, 
        confirming that $6$ is composite. 
        Its factorization is $2 \cdot 3$.
  \item Tick to $n=7$:  
        \[
        r_2(7) = 1, \quad r_3(7) = 1.
        \]
        No counter is zero.  
        Therefore no smaller prime divides 7, 
        so $7$ is prime.  
        We append $7$ to $P$ and record $\sigma(7) = (0,0,0,1,0,\dots)$.
\end{enumerate}

\begin{studentexample}
This illustrates the odometer’s rhythm:
\begin{itemize}
  \item At $n=6$, residues wrapped to zero, producing a composite.
  \item At $n=7$, no residue wrapped, forcing a new axis.  
        Thus 7 emerges as prime.
\end{itemize}
\end{studentexample}

\subsection{p-adic Carry Depth}

The odometer rule can also be described in terms of \emph{$p$-adic valuation}, 
which measures the depth of a carry.

\begin{definition}[p-adic carry depth]
For a prime $p$, define
\[
d_p(n) = \max \{ t \geq 0 : n \equiv -1 \pmod{p^t} \}.
\]
Then the valuation of $p$ in $n+1$ is
\[
\nu_p(n+1) = d_p(n).
\]
\end{definition}

\begin{studentexample}
In base-10, the number of trailing 9’s in $n$ equals the carry depth when incrementing $n$.  
Similarly, in base-$p$, the number of trailing $(p-1)$ digits controls the $p$-adic carry depth.  

For instance, $n=8 \equiv -1 \pmod{9}$, so $d_3(8)=2$.  
Thus $\nu_3(9)=2$, consistent with $9 = 3^2$.
\end{studentexample}

\subsection*{Exercises (Set \#3)}

\begin{studentexercise}
At $n=10$, with $P = \{2,3,5,7\}$, compute the residues $r_p(10)$ for each $p \in P$.  
Advance the odometer to $n=11$.  
Which case occurs: composite or prime?
\end{studentexercise}

\begin{studentexercise}
Verify that $14$ is composite by tracking residues with $P=\{2,3,5\}$.  
Then advance to $15$ and show that it is also composite.  
What happens at $n=16$ and $n=17$?
\end{studentexercise}

\begin{studentexercise}
Use the definition of $p$-adic carry depth to compute $\nu_2(31)$, 
given that $31 \equiv -1 \pmod{32}$.  
Check your result by factoring $32$.
\end{studentexercise}

\section{The Odometer Theorem}

The Prime Odometer does more than suggest primes emerge as carries: 
it can be proved to generate the correct prime exponent vector $\sigma(n)$ 
for every integer $n \geq 2$. 
This establishes the algorithm’s \emph{correctness}. 

\begin{theorem}[Prime Odometer Correctness]
Initialize at $n=2$ with
\[
\sigma(2) = (1,0,0,\dots), \quad P=\{2\}, \quad r_2(2)=0.
\]
At each successor step $n \mapsto n+1$, update residues as described in Section 4. 
Then:
\begin{enumerate}
  \item For all $n \geq 2$, the odometer state correctly represents $\sigma(n)$.
  \item If $n+1$ is composite, then $\sigma(n+1)$ is determined exactly by the residue wraps 
        of primes $p \leq \sqrt{n+1}$.
  \item If $n+1$ is prime, then $\sigma(n+1) = (0,0,\dots,0,1)$ is a one-hot vector on a new axis.
\end{enumerate}
Thus the odometer generates the exact prime exponent vector at every step 
and identifies primes without external verification.
\end{theorem}

\begin{proof}[Proof by induction]
\textbf{Base case.}  
At $n=2$, we initialize $\sigma(2)=(1,0,0,\dots)$, $P=\{2\}$, and $r_2(2)=0$.  
This is correct by definition.

\medskip
\noindent
\textbf{Inductive step.}  
Assume the odometer state is correct for $n$.  
Let $m = n+1$.  
Update each residue $r_p(n)$ to $r_p(m) = (r_p(n)+1) \bmod p$. 

\begin{itemize}
  \item \emph{Composite case.}  
        If some $p \in P$ with $p \leq \sqrt{m}$ has $r_p(m)=0$, 
        then $p \mid m$.  
        Repeated wraps reveal $\nu_p(m)$, and dividing out these prime powers 
        leaves either $1$ or a cofactor $q > \sqrt{m}$.  
        In the latter case, $q$ must itself be prime, so $m$ is composite but fully factorized.  
        Hence $\sigma(m)$ is computed exactly.
  \item \emph{Prime case.}  
        If no such $p$ divides $m$, then $m$ has no prime factor $\leq \sqrt{m}$.  
        Therefore $m$ is prime.  
        Append $m$ to $P$, initialize $r_m(m)=0$, and set $\sigma(m)$ to a one-hot vector.
\end{itemize}

In both cases, the state is updated consistently.  
Thus by induction, the odometer remains correct for all $n \geq 2$.
\end{proof}

\subsection{Illustrative Walkthroughs}

\begin{studentexample}
\textbf{Case 1: Composite.}  
At $n=8$, residues for $P=\{2,3,5,7\}$ are updated to give $r_2(9)=0$.  
Thus $2 \mid 9$? No — but check carefully: in fact $r_3(9)=0$, so $3 \mid 9$.  
Repeatedly checking gives $\nu_3(9)=2$, confirming $9=3^2$.  
The odometer records $\sigma(9)=(0,2,0,0,\dots)$.
\end{studentexample}

\begin{studentexample}
\textbf{Case 2: Prime.}  
At $n=10$, the successor is $m=11$.  
Residues for $2,3,5,7$ are all nonzero, so no small prime divides 11.  
Therefore $11$ is prime, and $\sigma(11) = (0,0,0,0,1,0,\dots)$.
\end{studentexample}

\subsection{Corollary: Canonical Carry Certificate}

\begin{corollary}
For every $k \geq 1$, the step
\[
p_{k+1}-1 \in S_k, \qquad p_{k+1} \notin S_k
\]
forces $\sigma(p_{k+1}) = (0,0,\dots,0,1)$.  
That is, the primality of $p_{k+1}$ is certified structurally by the canonical carry, 
with no need for external tests.
\end{corollary}

\begin{studentnote}
This is one of the most striking features of the odometer: 
\emph{primes prove themselves}.  
Their structure is forced by the failure of redistribution, 
just as the carry to a new digit is forced in positional systems.
\end{studentnote}

\subsection*{Exercises (Set \#4)}

\begin{studentexercise}
Use the induction argument to show that the odometer produces the correct factorization 
for $n=12$. Verify by hand that $\sigma(12)=(2,1,0,0,\dots)$.
\end{studentexercise}

\begin{studentexercise}
Trace the odometer from $n=14$ to $n=17$.  
Which are composites? Which are prime?  
Write the corresponding $\sigma(n)$ vectors.
\end{studentexercise}

\begin{studentexercise}
Explain why it is enough to check divisibility by primes $\leq \sqrt{m}$ 
when deciding if $m$ is prime. 
(Hint: what would happen if $m$ had a factor larger than $\sqrt{m}$?)
\end{studentexercise}

\section{Comparison with Classical Sieves}

The Prime Odometer provides a constructive, sequential way of generating primes. 
To appreciate its novelty, it is helpful to compare it with the most famous 
classical method: the Sieve of Eratosthenes. 

\subsection{The Sieve of Eratosthenes}

Devised over two thousand years ago, the sieve remains a cornerstone algorithm 
for enumerating primes up to a given bound $N$:

\begin{enumerate}
  \item Write down the list $\{2,3,\dots,N\}$.
  \item Select the next unused number $p$; it is prime.
  \item Eliminate all multiples of $p$ from the list.
  \item Repeat until all numbers are marked as composite or declared prime.
\end{enumerate}

This process is \emph{eliminative}: primes are discovered only after composites 
have been systematically removed. The sieve is efficient --- its complexity is 
$O(N \log\log N)$ --- but conceptually it frames primes as ``leftovers.''

\subsection{The Prime Odometer}

By contrast, the Prime Odometer is \emph{generative}. 
It advances $n \mapsto n+1$ sequentially, updating residue counters. 
\begin{itemize}
  \item If residues wrap to zero, $n+1$ is composite, and its factorization 
        follows directly from the counters.
  \item If no counter wraps, redistribution fails and a new axis is forced. 
        Thus $n+1$ emerges as prime.
\end{itemize}

In other words:
\begin{itemize}
  \item The sieve says: ``Remove composites, and what remains is prime.''
  \item The odometer says: ``Advance step by step; primes are inevitable 
        carry events.''
\end{itemize}

\subsection{Side-by-Side Comparison}

The following table illustrates the difference for numbers $2 \leq n \leq 20$.

\begin{center}
\begin{tabular}{@{}lll@{}}
\toprule
$n$ & \textbf{Sieve of Eratosthenes} & \textbf{Prime Odometer} \\
\midrule
2   & Initial prime, multiples removed & First prime axis, $\sigma(2)=(1,0,0,\dots)$ \\
3   & Declared prime, remove multiples & New axis forced, $\sigma(3)=(0,1,0,\dots)$ \\
4   & Marked composite ($2 \cdot 2$)   & Residue wrap: divisible by 2 \\
5   & Declared prime, remove multiples & New axis forced, $\sigma(5)=(0,0,1,\dots)$ \\
6   & Marked composite ($2 \cdot 3$)   & Residue wraps for 2 and 3 \\
7   & Declared prime, remove multiples & New axis forced, $\sigma(7)=(0,0,0,1,\dots)$ \\
$\vdots$ & $\vdots$ & $\vdots$ \\
19  & Declared prime                   & New axis forced, $\sigma(19)=(0,0,0,0,0,0,1,\dots)$ \\
20  & Marked composite ($2 \cdot 2 \cdot 5$) & Residue wraps: divisible by 2 and 5 \\
\bottomrule
\end{tabular}
\end{center}

\subsection{Conceptual Contrast}

\begin{studentnote}
\begin{itemize}
  \item \textbf{Sieve:} primes are what survive elimination.
  \item \textbf{Odometer:} primes are what emerge when redistribution fails.
\end{itemize}
\end{studentnote}

This is the heart of the difference. 
Both methods are correct, but they offer very different mental models. 
The sieve emphasizes absence; the odometer emphasizes inevitability.

\subsection*{Exercises (Set \#5)}

\begin{studentexercise}
Run the sieve of Eratosthenes for $N=30$.  
Which numbers are marked composite? Which are declared prime?  
Compare with the odometer’s description of the same interval.
\end{studentexercise}

\begin{studentexercise}
Explain in your own words the difference between an \emph{eliminative} 
and a \emph{generative} algorithm.  
Give one example of each outside of number theory.
\end{studentexercise}

\begin{studentexercise}
Why does the sieve require $O(N)$ storage, 
while the odometer requires only $O(\pi(\sqrt{N}))$ residue counters?  
(Hint: consider how many primes are $\leq \sqrt{N}$.)
\end{studentexercise}

\section{Implications and Perspectives}

The Prime Odometer reframes the nature of primality.  
Instead of treating primes as mysterious anomalies scattered along the number line, 
it casts them as \emph{inevitable structural events}.  
This perspective has several important consequences, both mathematical and philosophical.

\subsection{Positive Definition of Primes}

Classically:
\begin{quote}
``A prime is an integer greater than one that has no divisors other than 1 and itself.''
\end{quote}

This is a \emph{negative definition}: it defines primes by what they lack.  

In the odometer framework, however, we can say:
\begin{quote}
``A prime is the one-hot successor state forced when redistribution across existing prime axes fails.''
\end{quote}

This is a \emph{positive definition}: it specifies what primes \emph{are}, not merely what they are not.  
Primes emerge constructively as canonical carries, rather than being discovered by absence.

\subsection{Inductive Certification}

Because the odometer maintains residues for all primes up to $\sqrt{n}$, 
the correctness of $\sigma(n+1)$ is guaranteed inductively.  
No external primality test is required.  

\begin{studentnote}
\textbf{Key idea:}  
Primality is not something to be checked; it is something to be \emph{forced}.  
If all counters fail, the system must create a new axis.  
This is an intrinsic certificate of primality.
\end{studentnote}

\subsection{Duality with Positional Systems}

There is a striking parallel between positional base-$B$ systems and PE-space:

\begin{itemize}
  \item In base-$B$, carries occur constantly.  
        Every time we increment a string of trailing $(B-1)$’s, 
        we must create a carry into the next digit.  
        Example: $999 \mapsto 1000$.
  \item In PE-space, carries occur sparsely.  
        They happen exactly at primes, with frequency about $1 / \log n$ 
        (by the Prime Number Theorem).
\end{itemize}

Thus:
\[
\text{Base-$B$: constant-rate carries} \qquad 
\longleftrightarrow \qquad
\text{PE-space: sparse canonical carries}.
\]

\begin{studentexample}
Think of base-2 (binary). Every second number requires a carry: $1 \mapsto 10$, $11 \mapsto 100$, etc.  
By contrast, in PE-space the next carry may be very far away: 
after 101, the next prime 103 may require 2 steps, 
but after 113, the next prime 127 may require 14 steps.  
The carries are irregular but inevitable.
\end{studentexample}

\subsection{Broader Implications}

The odometer suggests new ways of teaching and thinking about primes:

\begin{itemize}
  \item \textbf{Pedagogical clarity.}  
        Students can view primes as positive milestones of arithmetic, 
        not as puzzles that must be checked by trial division.
  \item \textbf{Analytic connection.}  
        The fact that carries appear at rate $\sim 1/\log n$ 
        ties the odometer directly to the Prime Number Theorem.
  \item \textbf{Philosophical shift.}  
        Primes are not anomalies; they are structural necessities.  
        They mark the places where arithmetic must expand.
\end{itemize}

\begin{studentnote}
\textbf{Analogy:}  
In physics, a phase transition (like water freezing into ice) 
occurs when redistribution of energy fails.  
In number theory, a prime emerges when redistribution of factors fails.  
Both are examples of structural inevitability.
\end{studentnote}

\subsection*{Exercises (Set \#6)}

\begin{studentexercise}
State in your own words the difference between the \emph{negative definition} 
and the \emph{positive definition} of primes.  
Why might the positive definition be easier to teach to beginners?
\end{studentexercise}

\begin{studentexercise}
Explain the analogy between base-$B$ carries and PE-space carries.  
Why are positional carries constant in frequency, while prime carries become sparser?
\end{studentexercise}

\begin{studentexercise}
Primes occur with asymptotic density $\sim 1 / \log n$.  
Use this fact to argue why canonical carries in PE-space become rarer but never disappear. 
(Hint: recall Euclid’s proof that there are infinitely many primes.)
\end{studentexercise}

\section{Exercises and Projects}

Throughout the paper, we have introduced short exercise sets at the end of each section.  
Here we gather them together and add some extended projects.  
These are intended to reinforce understanding and to encourage exploration 
beyond the formal text.

\subsection{Collected Exercises}

\begin{enumerate}[label=\textbf{Set \#\arabic*:}, leftmargin=*]

  \item \textbf{Prime Exponent Space}
  \begin{enumerate}
    \item Compute $\sigma(20), \sigma(30), \sigma(64), \sigma(210)$.  
    \item Show that $\sigma(45) + \sigma(14) = \sigma(630)$.  
    \item Which numbers less than 20 correspond to one-hot vectors? Why?
  \end{enumerate}

  \item \textbf{Smooth Frontiers and Pre-Primes}
  \begin{enumerate}
    \item List all elements of $S_2$ up to 30. Identify pre-primes and their successor primes.  
    \item Show that 30 is a pre-prime in $S_3$. What prime follows it? Write both $\sigma(30)$ and $\sigma(31)$.  
    \item Verify that for each prime $p$, $p-1 \in S_k$ for appropriate $k$.  
  \end{enumerate}

  \item \textbf{Prime Odometer Algorithm}
  \begin{enumerate}
    \item Starting at $n=10$, with $P=\{2,3,5,7\}$, compute residues and decide if 11 is prime.  
    \item Track residues for $n=14,15,16,17$. Which are composites? Which is prime?  
    \item Use $p$-adic carry depth to show that $\nu_2(32)=5$.  
  \end{enumerate}

  \item \textbf{Odometer Theorem}
  \begin{enumerate}
    \item Use induction to show $\sigma(12)=(2,1,0,0,\dots)$.  
    \item Trace the odometer from $n=14$ to $n=17$. Write each $\sigma(n)$.  
    \item Explain why it is enough to test divisibility only by primes $\leq \sqrt{m}$.  
  \end{enumerate}

  \item \textbf{Comparison with Classical Sieves}
  \begin{enumerate}
    \item Run Eratosthenes’ sieve for $N=30$. Compare with odometer results.  
    \item Give an example of a generative algorithm and an eliminative algorithm outside number theory.  
    \item Explain why the sieve uses $O(N)$ storage while the odometer needs only $O(\pi(\sqrt{N}))$ counters.  
  \end{enumerate}

  \item \textbf{Implications and Perspectives}
  \begin{enumerate}
    \item Contrast the negative and positive definitions of primes. Which is pedagogically clearer?  
    \item Explain the analogy between positional carries and canonical carries.  
    \item Why do canonical carries get sparser but never vanish?  
  \end{enumerate}

\end{enumerate}

\subsection{Extended Projects}

\begin{enumerate}[label=\textbf{Project \#\arabic*:}, leftmargin=*]

  \item \textbf{Implementing the Prime Odometer.}  
  Write a short computer program (in Python, Java, or another language) 
  that generates primes using the residue update rule of the Prime Odometer.  
  Compare its performance with a basic implementation of the Sieve of Eratosthenes 
  for $N=1000$.  

  \item \textbf{Prime Gaps.}  
  Use your implementation to record prime gaps (the difference between consecutive primes).  
  Plot the first 200 prime gaps. Do you see any patterns?  
  How does this relate to the fact that carries in PE-space become sparser?  

  \item \textbf{Residue Visualizations.}  
  For a given prime $p$, track the residue counter $r_p(n)$ as $n$ advances.  
  Make a diagram showing how the counter cycles.  
  Overlay several primes (e.g. 2, 3, 5) to see how their cycles interact.  

  \item \textbf{Exploring Pre-Primes.}  
  Make a list of the first 50 pre-primes.  
  Do they cluster or show any patterns?  
  What can you say about the size of the prime gaps they precede?  

  \item \textbf{Philosophical Essay.}  
  Write a short essay comparing the eliminative vs. generative definitions of primes.  
  Reflect on how different perspectives can change the way we view mathematical truth.  

\end{enumerate}

\begin{studentnote}
\textbf{Advice for students:}  
Do not treat exercises as a checklist.  
The goal is to develop intuition.  
When possible, work with small numbers, draw diagrams, 
and think about analogies to familiar systems like decimal odometers. 
\end{studentnote}

\section{Conclusion}

We have traveled from the classical negative definition of primes to a new, 
constructive viewpoint: primes as \emph{canonical carries} in Prime Exponent Space.  
This journey highlights a central theme:

\begin{quote}
Primes are not random survivors of elimination, 
but inevitable milestones of arithmetic --- 
the moments when redistribution fails and a new axis must be created.
\end{quote}

The Prime Odometer embodies this idea algorithmically.  
By maintaining residue counters and advancing one tick at a time, 
it generates the prime exponent vector $\sigma(n)$ for every integer $n$.  
At composites, residues wrap and factors are revealed.  
At primes, no residue wraps, and a one-hot vector is forced.  
Primality is therefore certified structurally, without external tests.

\subsection*{Comparison with Classical Approaches}

The contrast with the Sieve of Eratosthenes is striking:
\begin{itemize}
  \item The sieve is \emph{eliminative}: it removes composites to expose primes.  
  \item The odometer is \emph{generative}: it constructs successors, 
        and primes emerge when carrying is unavoidable.
\end{itemize}

Both methods are valid, but they offer different conceptual models.  
The sieve emphasizes absence; the odometer emphasizes inevitability.  

\subsection*{Pedagogical Significance}

For students, the odometer provides:
\begin{itemize}
  \item A positive, constructive definition of primality.  
  \item A step-by-step algorithm that can be simulated by hand or computer.  
  \item A unifying analogy to positional number systems: 
        primes as the ``carry digits'' of arithmetic.  
\end{itemize}

By reframing primes in this way, we open doors to intuition, teaching clarity, 
and interdisciplinary analogies.  

\begin{studentnote}
\textbf{Key takeaway:}  
The mystery of primes is not that they appear irregularly, 
but that they appear inevitably.  
They are the canonical carries of arithmetic --- sparse, structural, 
and unavoidable.  
\end{studentnote}

\appendix

\section*{Appendix A: Minimal Pseudocode for the Prime Odometer}

This pseudocode shows how to implement the Prime Odometer in a programming language 
such as Python. It is deliberately minimal, focusing only on primality detection 
rather than full factorization.

\begin{verbatim}
# Prime Odometer (minimal version)

# Initialize
P = [2]                 # list of known primes
r = {2: 0}              # residues modulo each prime
n = 2                   # starting integer

def tick(n, P, r):
    m = n + 1
    composite = False

    # 1. Increment residues
    for p in P:
        r[p] = (r[p] + 1) % p

    # 2. Check for divisibility
    for p in P:
        if p * p > m: 
            break
        if r[p] == 0:
            composite = True
            break

    if composite:
        return (m, "composite", P, r)
    else:
        # m is prime
        P.append(m)
        r[m] = 0
        return (m, "prime", P, r)

# Driver to enumerate primes up to N
N = 50
while n < N:
    (m, tag, P, r) = tick(n, P, r)
    if tag == "prime":
        print(m)
    n = m
\end{verbatim}

\begin{studentnote}
This code prints all primes up to $N=50$.  
Students are encouraged to extend it to recover factorizations of composites, 
or to visualize the behavior of residue counters over time.  
\end{studentnote}

\section*{Appendix B: Extended Notation Reference}

For convenience, we collect the key symbols used throughout the paper.

\begin{center}
\begin{tabular}{ll}
\toprule
Symbol & Meaning \\
\midrule
$\backslash$ & Set difference; $A \backslash B$ = elements of $A$ not in $B$ \\
$\equiv$ & Congruence modulo $p$; $a \equiv b \pmod{p}$ means $p \mid (a-b)$ \\
$\sim$ & Asymptotic equivalence; $f(n) \sim g(n)$ if $f/g \to 1$ as $n \to \infty$ \\
$\forall$ & ``For all'' \\
$\exists$ & ``There exists'' \\
$\mathbb{N}_{\geq 2}$ & Integers $\{2,3,4,\dots\}$ \\
$\sigma(n)$ & Prime exponent vector of $n$ \\
$e_i(n)$ & Exponent of prime $p_i$ in factorization of $n$ \\
$\nu_p(n)$ & $p$-adic valuation; highest power of $p$ dividing $n$ \\
$d_p(n)$ & $p$-adic carry depth at $n$ \\
$S_k$ & $p_k$-smooth semigroup; integers with prime factors $\leq p_k$ \\
Pre-prime & A number $n \in S_k$ with $n+1 \notin S_k$ \\
Canonical carry & Step $p_{k+1}-1 \mapsto p_{k+1}$ introducing new axis \\
$P$ & List of known primes in odometer state \\
$r_p(n)$ & Residue of $n$ modulo $p$ \\
\bottomrule
\end{tabular}
\end{center}

\section*{References}

\begin{enumerate}
  \item G.H. Hardy and E.M. Wright. \emph{An Introduction to the Theory of Numbers}.  
        Oxford University Press, 6th ed., 2008.
  \item R. Crandall and C. Pomerance. \emph{Prime Numbers: A Computational Perspective}.  
        Springer, 2nd ed., 2005.
  \item H. Cohen. \emph{Number Theory, Volume I: Tools and Diophantine Equations}.  
        Springer, 2nd ed., 2007.
\end{enumerate}
\end{document}
